\documentclass[12pt, oneside, openany]{scrreprt}
\usepackage[a4paper, margin=3cm]{geometry}

% Citation style is specified in "jurzitat.csl" file. To make this work you need to compile this file using the following recipe
% 1. Compiling with latex/pdflatex/xelatex, 2. run the citeproc-lua on this file, 3. follow up one or two more runs of latex/pdflatex/xelatex compilation.
\usepackage{citation-style-language}
\cslsetup{style = jurzitat}
\addbibresource{literatur.bib} % Your .bib file

\usepackage{csquotes} 
\usepackage{hyperref} 
\usepackage{microtype} 
\usepackage[ngerman]{babel}
\setlength{\parindent}{0pt}
\usepackage{parskip}

\usepackage{titlesec}

\setcounter{secnumdepth}{5} % Gliederungsebenen

% **Kapitel = Kapitel 1, Kapitel 2, ...**
\renewcommand{\thechapter}{Kapitel \arabic{chapter}}

% **Abschnitte = A, B, C...**
\renewcommand{\thesection}{\Alph{section}.}

% **Unterabschnitte = I, II, III...**
\renewcommand{\thesubsection}{\Roman{subsection}.}

% **Subsubsections = 1, 2, 3...**
\renewcommand{\thesubsubsection}{\arabic{subsubsection}.}

% **Paragraphen = a, b, c...**
\renewcommand{\theparagraph}{\alph{paragraph}.}

% **Subparagraphen = (1), (2), (3)...**
\renewcommand{\thesubparagraph}{(\arabic{subparagraph})}

% Titelstile definieren
\titleformat{\chapter}[display]
  {\bfseries\LARGE} % Kapitel fett & groß
  {\thechapter}{1ex}{\titlerule\vspace{1ex}\Huge}

\titleformat{\section}
  {\large\bfseries} % Fett & groß
  {\thesection}{1ex}{}

\titleformat{\subsection}
  {\large\bfseries} % Mittelgroß
  {\thesubsection}{1ex}{}

\titleformat{\subsubsection}
  {\normalsize\bfseries} % Klein
  {\thesubsubsection}{1ex}{}

\titleformat{\paragraph}
  {\normalsize\itshape} % Kursiv
  {\theparagraph}{1ex}{}

\titleformat{\subparagraph}
  {\normalsize\itshape} % Noch kleiner & kursiv
  {\thesubparagraph}{1ex}{}



\usepackage{tocloft}
\setlength{\cftchapnumwidth}{6em}
%\setlength{\cftsecindent}{0pt}
\setlength{\cftsecnumwidth}{3em}
%\setlength{\cftsubsecindent}{0pt}
\setlength{\cftsubsecnumwidth}{3em}
%\setlength{\cftsubsubsecindent}{0pt}
\setlength{\cftsubsubsecnumwidth}{3em}
%\setlength{\cftparaindent}{0pt}
\setlength{\cftparanumwidth}{3em}
%\setlength{\cftsubparaindent}{0pt}
\setlength{\cftsubparanumwidth}{3em}



\begin{document}

\title{Titel der Arbeit}
\subtitle{Untertitel der Arbeit}
\author{Vorname Nachname}
\date{Datum der Abgabe}

\maketitle

\setcounter{tocdepth}{6}
\tableofcontents

\chapter{Einleitung}

\section{Einführung}
Hello, world!

\section{Nächster Abschnitt}

So kannst du bequem zitieren: \enquote{Pacta \enquote{sunt} servanda.}

Jetzt folgt --- ein --- Gedankenstrich.

Jetzt kommt ein \emph{kursiv} geschriebenes Wort.

Mit dem nächsten Satz beginnt eine neue Seite \newpage

\include{content/extern}

\chapter{Das zweite Kapitel}

Jetzt testen wir Fußnoten für jeden Beitragstyp.

Urteil.\cite[prefix = {}, suffix = {.}]{UrteilVom160520132013}

Buch/Monographie.\cite[prefix = {}, suffix = {.}, page = 23]{buscheVerwaltungsautomation20Automatisiert2023}

Beitrag in Zeitschrift.\cite[prefix = {}, suffix = {.}, page = 750]{martiniOnceOnlyOnly2017}

Beitrag in Sammelband.\cite[prefix = {}, suffix = {.}, page = 199]{druenAmtsermittlungsgrundsatzUndRisikomanagement2019a}

Beitrag in Kommentar.\cite[prefix = {}, suffix = { Rn. 88.}]{klein882024} \cite[prefix = {}, suffix = {Rn. 102.}]{difabioArt22025}

Webseite.\cite[prefix = {}, suffix = {.}]{kochMerzWerDaten2025}

Drucksache.\cite[prefix = {}, suffix = {.}]{StellungnahmeEntwurfGesetzes}

Scheint zu funktionieren. Jetzt kommen die übrigen Gliederungsebenen.

\section{Das ist eine Sektion(Ebene 2) }

\subsection{Das ist eine Subsektion (Ebene 3)}

\subsubsection{Jetzt eine Subsubsektion (Ebene 4)}

\paragraph{Das nennt man einen Paragraphen (Ebene 5)}

\subparagraph{Folglich handelt es sich hier um einen Subparagraphen (Ebene 6)}

Auf der nächsten Seite generieren wir ein Literaturverzeichnis.

\printbibliography[title=Literaturverzeichnis, nottype = {legal_case}] 

\end{document}